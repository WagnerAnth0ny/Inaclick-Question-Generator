\documentclass{article}
\usepackage{amsmath}
\begin{document}

\paragraph{Questão 1:}
{Em um triângulo retângulo, o ângulo θ é agudo e sen θ = 3/5. Quais são os valores de cos θ e tg θ, respectivamente?}

\vspace{\baselineskip}Opções:

\vspace{\baselineskip}\textbf{A)}: 4/5 e 3/4 

\textbf{B)}: 5/4 e 3/4 

\textbf{C)}: 3/4 e 4/3 

\textbf{D)}: 4/3 e 5/3 

\textbf{E)}: 3/5 e 4/5 

\vspace{\baselineskip}\textbf{Resposta Correta}: A)

\vspace{\baselineskip}\textbf{Dica}: Use a identidade trigonométrica sin²θ + cos²θ = 1 para encontrar cos θ.

\paragraph{Questão 2:}
{Considerando um triângulo retângulo, onde θ é um ângulo agudo e tg θ = 7/24. Quais são os valores de sen θ e cos θ, respectivamente?}

\vspace{\baselineskip}Opções:

\vspace{\baselineskip}\textbf{A)}: 7/25 e 24/25 

\textbf{B)}: 25/24 e 7/25 

\textbf{C)}: 24/25 e 7/25 

\textbf{D)}: 24/25 e 25/7 

\textbf{E)}: 7/24 e 1 

\vspace{\baselineskip}\textbf{Resposta Correta}: A)

\vspace{\baselineskip}\textbf{Dica}: Use o teorema de Pitágoras em um triângulo com catetos 7 e 24 para encontrar a hipotenusa.

\paragraph{Questão 3:}
{Em um triângulo retângulo, x é um ângulo agudo tal que sen x = 5/13. Quais são os valores de cos x e tg x, respectivamente?}

\vspace{\baselineskip}Opções:

\vspace{\baselineskip}\textbf{A)}: 5/12 e 13/12 

\textbf{B)}: 12/13 e 5/12 

\textbf{C)}: 12/5 e 13/5 

\textbf{D)}: 5/13 e 12/13 

\textbf{E)}: 5/13 e 13/5 

\vspace{\baselineskip}\textbf{Resposta Correta}: B)

\vspace{\baselineskip}\textbf{Dica}: Construa um triângulo utilizando os comprimentos dados e encontre o cateto adjacente usando o teorema de Pitágoras.

\paragraph{Questão 4:}
{Sabendo que em um triângulo retângulo o ângulo θ satisfaz cos θ = 8/17. Quais são os valores de sen θ e tg θ, respectivamente?}

\vspace{\baselineskip}Opções:

\vspace{\baselineskip}\textbf{A)}: 15/17 e 8/15 

\textbf{B)}: 17/15 e 15/8 

\textbf{C)}: 8/15 e 17/8 

\textbf{D)}: 8/17 e 15/8 

\textbf{E)}: 15/8 e 17 

\vspace{\baselineskip}\textbf{Resposta Correta}: A)

\vspace{\baselineskip}\textbf{Dica}: A hipotenusa é 17. Use o teorema de Pitágoras para achar o cateto oposto, que é 15.

\paragraph{Questão 5:}
{No triângulo retângulo, θ é um ângulo agudo e sen θ = 9/41. Quais são os valores de cos θ e tg θ, respectivamente?}

\vspace{\baselineskip}Opções:

\vspace{\baselineskip}\textbf{A)}: 40/41 e 9/40 

\textbf{B)}: 41/40 e 9/41 

\textbf{C)}: 40/9 e 41/9 

\textbf{D)}: 9/40 e 41/40 

\textbf{E)}: 9/41 e 40/41 

\vspace{\baselineskip}\textbf{Resposta Correta}: A)

\vspace{\baselineskip}\textbf{Dica}: Construa um triângulo retângulo com catetos 9 e 40, e hipotenusa 41.

\paragraph{Questão 6:}
{Em um triângulo retângulo, x é um ângulo agudo e tg x = 5/12. Quais são os valores de sen x e cos x, respectivamente?}

\vspace{\baselineskip}Opções:

\vspace{\baselineskip}\textbf{A)}: 5/13 e 12/13 

\textbf{B)}: 12/13 e 5/13 

\textbf{C)}: 13/5 e 13/12 

\textbf{D)}: 5/12 e 5/13 

\textbf{E)}: 12/5 e 13 

\vspace{\baselineskip}\textbf{Resposta Correta}: A)

\vspace{\baselineskip}\textbf{Dica}: Use a relação entre os catetos e a hipotenusa de um triângulo retângulo para achar os valores.

\paragraph{Questão 7:}
{Em um triângulo retângulo, θ é um ângulo agudo com cos θ = 6/10. Quais são os valores de sen θ e tg θ, respectivamente?}

\vspace{\baselineskip}Opções:

\vspace{\baselineskip}\textbf{A)}: 4/5 e 3/4 

\textbf{B)}: 3/5 e 4/3 

\textbf{C)}: 8/10 e 6/8 

\textbf{D)}: 8/10 e 6/8 

\textbf{E)}: 4/10 e 5/6 

\vspace{\baselineskip}\textbf{Resposta Correta}: A)

\vspace{\baselineskip}\textbf{Dica}: Você pode simplificar a fração 6/10 antes de aplicar as identidades trigonométricas.

\paragraph{Questão 8:}
{Considerando um triângulo retângulo em que o ângulo x é agudo e cos x = 15/17. Quais são os valores de sen x e tg x, respectivamente?}

\vspace{\baselineskip}Opções:

\vspace{\baselineskip}\textbf{A)}: 8/17 e 15/8 

\textbf{B)}: 17/8 e 8/15 

\textbf{C)}: 15/17 e 15/8 

\textbf{D)}: 8/17 e 8/15 

\textbf{E)}: 8/17 e 17/15 

\vspace{\baselineskip}\textbf{Resposta Correta}: A)

\vspace{\baselineskip}\textbf{Dica}: A hipotenusa é 17. Considere usar Pitágoras para encontrar o cateto oposto.

\paragraph{Questão 9:}
{Sabendo que em um triângulo retângulo, o ângulo θ é agudo e sen θ = 7/25. Quais são os valores de cos θ e tg θ, respectivamente?}

\vspace{\baselineskip}Opções:

\vspace{\baselineskip}\textbf{A)}: 24/25 e 7/24 

\textbf{B)}: 25/24 e 24/7 

\textbf{C)}: 24/7 e 25/24 

\textbf{D)}: 7/24 e 25/7 

\textbf{E)}: 7/25 e 24/25 

\vspace{\baselineskip}\textbf{Resposta Correta}: A)

\vspace{\baselineskip}\textbf{Dica}: Use o teorema de Pitágoras com hipotenusa 25 para encontrar o cateto adjacente que é 24.

\paragraph{Questão 10:}
{Em um triângulo retângulo, x é um ângulo agudo e tg x = 3/4. Quais são os valores de sen x e cos x, respectivamente?}

\vspace{\baselineskip}Opções:

\vspace{\baselineskip}\textbf{A)}: 3/5 e 4/5 

\textbf{B)}: 5/4 e 3/5 

\textbf{C)}: 4/3 e 5/3 

\textbf{D)}: 3/4 e 5/4 

\textbf{E)}: 3/5 e 5/4 

\vspace{\baselineskip}\textbf{Resposta Correta}: A)

\vspace{\baselineskip}\textbf{Dica}: Aplique o teorema de Pitágoras para encontrar a hipotenusa e então determinar sen x e cos x.

\paragraph{Questão 11:}
{Considere um triângulo retângulo com catetos medindo 6 cm e 8 cm. Determine a hipotenusa do triângulo.}

\vspace{\baselineskip}Opções:

\vspace{\baselineskip}\textbf{a) 10 cm}: 10 cm 

\textbf{b) 9 cm}: 9 cm 

\textbf{c) 12 cm}: 12 cm 

\textbf{d) 14 cm}: 14 cm 

\textbf{e) 15 cm}: 15 cm 

\vspace{\baselineskip}\textbf{Resposta Correta}: 10 cm

\vspace{\baselineskip}\textbf{Dica}: Utilize o teorema de Pitágoras: a soma dos quadrados dos catetos é igual ao quadrado da hipotenusa.

\paragraph{Questão 12:}
{Em um triângulo retângulo, os catetos medem 9 cm e 12 cm. Qual é a medida da hipotenusa?}

\vspace{\baselineskip}Opções:

\vspace{\baselineskip}\textbf{a) 15 cm}: 15 cm 

\textbf{b) 18 cm}: 18 cm 

\textbf{c) 21 cm}: 21 cm 

\textbf{d) 14 cm}: 14 cm 

\textbf{e) 22 cm}: 22 cm 

\vspace{\baselineskip}\textbf{Resposta Correta}: 15 cm

\vspace{\baselineskip}\textbf{Dica}: Lembre-se de que a hipotenusa é calculada através da raiz quadrada da soma dos quadrados dos catetos.

\paragraph{Questão 13:}
{Um triângulo retângulo tem catetos medindo 5 cm e 12 cm. Determine a hipotenusa desse triângulo.}

\vspace{\baselineskip}Opções:

\vspace{\baselineskip}\textbf{a) 13 cm}: 13 cm 

\textbf{b) 14 cm}: 14 cm 

\textbf{c) 15 cm}: 15 cm 

\textbf{d) 16 cm}: 16 cm 

\textbf{e) 17 cm}: 17 cm 

\vspace{\baselineskip}\textbf{Resposta Correta}: 13 cm

\vspace{\baselineskip}\textbf{Dica}: Aplicar o teorema de Pitágoras pode ajudar na solução desta questão.

\paragraph{Questão 14:}
{No triângulo retângulo, os catetos são de 7 cm e 24 cm. Calcule a hipotenusa.}

\vspace{\baselineskip}Opções:

\vspace{\baselineskip}\textbf{a) 25 cm}: 25 cm 

\textbf{b) 24 cm}: 24 cm 

\textbf{c) 23 cm}: 23 cm 

\textbf{d) 26 cm}: 26 cm 

\textbf{e) 28 cm}: 28 cm 

\vspace{\baselineskip}\textbf{Resposta Correta}: 25 cm

\vspace{\baselineskip}\textbf{Dica}: Use a fórmula c² = a² + b², onde c é a hipotenusa, para encontrar a resposta.

\paragraph{Questão 15:}
{Dado um triângulo retângulo com catetos de 8 cm e 15 cm, encontre a hipotenusa.}

\vspace{\baselineskip}Opções:

\vspace{\baselineskip}\textbf{a) 17 cm}: 17 cm 

\textbf{b) 19 cm}: 19 cm 

\textbf{c) 20 cm}: 20 cm 

\textbf{d) 21 cm}: 21 cm 

\textbf{e) 22 cm}: 22 cm 

\vspace{\baselineskip}\textbf{Resposta Correta}: 17 cm

\vspace{\baselineskip}\textbf{Dica}: Teorema de Pitágoras é essencial para resolver problemas envolvendo hipotenusas.

\paragraph{Questão 16:}
{Calcule a hipotenusa de um triângulo retângulo cujos catetos são 3 cm e 4 cm.}

\vspace{\baselineskip}Opções:

\vspace{\baselineskip}\textbf{a) 5 cm}: 5 cm 

\textbf{b) 6 cm}: 6 cm 

\textbf{c) 7 cm}: 7 cm 

\textbf{d) 8 cm}: 8 cm 

\textbf{e) 9 cm}: 9 cm 

\vspace{\baselineskip}\textbf{Resposta Correta}: 5 cm

\vspace{\baselineskip}\textbf{Dica}: Esta combinação de medidas é um clássico exemplo das proporções pitagóricas.

\paragraph{Questão 17:}
{Considere um triângulo retângulo com catetos de medidas 9 cm e 40 cm. Qual é a hipotenusa?}

\vspace{\baselineskip}Opções:

\vspace{\baselineskip}\textbf{a) 41 cm}: 41 cm 

\textbf{b) 42 cm}: 42 cm 

\textbf{c) 43 cm}: 43 cm 

\textbf{d) 44 cm}: 44 cm 

\textbf{e) 45 cm}: 45 cm 

\vspace{\baselineskip}\textbf{Resposta Correta}: 41 cm

\vspace{\baselineskip}\textbf{Dica}: Aplique o teorema que envolve quadrados dos catetos para obter a resposta correta.

\paragraph{Questão 18:}
{Determine a hipotenusa de um triângulo retângulo com catetos de 11 cm e 60 cm.}

\vspace{\baselineskip}Opções:

\vspace{\baselineskip}\textbf{a) 61 cm}: 61 cm 

\textbf{b) 62 cm}: 62 cm 

\textbf{c) 63 cm}: 63 cm 

\textbf{d) 64 cm}: 64 cm 

\textbf{e) 65 cm}: 65 cm 

\vspace{\baselineskip}\textbf{Resposta Correta}: 61 cm

\vspace{\baselineskip}\textbf{Dica}: Tenha em mente que a relação entre catetos e hipotenusa segue sempre o teorema de Pitágoras.

\paragraph{Questão 19:}
{Em um triângulo retângulo, se os catetos medem 20 cm e 21 cm, qual será a hipotenusa?}

\vspace{\baselineskip}Opções:

\vspace{\baselineskip}\textbf{a) 29 cm}: 29 cm 

\textbf{b) 28 cm}: 28 cm 

\textbf{c) 27 cm}: 27 cm 

\textbf{d) 26 cm}: 26 cm 

\textbf{e) 25 cm}: 25 cm 

\vspace{\baselineskip}\textbf{Resposta Correta}: 29 cm

\vspace{\baselineskip}\textbf{Dica}: Calcular a soma dos quadrados dos catetos é fundamental para encontrar a hipotenusa.

\paragraph{Questão 20:}
{No triângulo retângulo cujos catetos medem 12 cm e 35 cm, encontre a hipotenusa.}

\vspace{\baselineskip}Opções:

\vspace{\baselineskip}\textbf{a) 37 cm}: 37 cm 

\textbf{b) 38 cm}: 38 cm 

\textbf{c) 39 cm}: 39 cm 

\textbf{d) 40 cm}: 40 cm 

\textbf{e) 41 cm}: 41 cm 

\vspace{\baselineskip}\textbf{Resposta Correta}: 37 cm

\vspace{\baselineskip}\textbf{Dica}: O conhecimento do teorema de Pitágoras é necessário para resolver este tipo de problema.

\paragraph{Questão 21:}
{Se os catetos de um triângulo retângulo medem 3b e 4b, então a tangente do ângulo oposto ao cateto maior é:}

\vspace{\baselineskip}Opções:

\vspace{\baselineskip}\textbf{A) \frac{3}{4}}: 3/4 

\textbf{B) \frac{4}{3}}: 4/3 

\textbf{C) \frac{3}{5}}: 3/5 

\textbf{D) \frac{4}{5}}: 4/5 

\textbf{E) \frac{5}{3}}: 5/3 

\vspace{\baselineskip}\textbf{Resposta Correta}: A

\vspace{\baselineskip}\textbf{Dica}: A tangente é calculada pela divisão do cateto oposto pelo cateto adjacente.

\paragraph{Questão 22:}
{Em um triângulo retângulo, se a hipotenusa mede 13c e um dos catetos mede 5c, então a tangente do ângulo oposto ao cateto desconhecido é:}

\vspace{\baselineskip}Opções:

\vspace{\baselineskip}\textbf{A) \frac{12}{5}}: 12/5 

\textbf{B) \frac{5}{12}}: 5/12 

\textbf{C) \frac{12}{13}}: 12/13 

\textbf{D) \frac{5}{13}}: 5/13 

\textbf{E) \frac{13}{5}}: 13/5 

\vspace{\baselineskip}\textbf{Resposta Correta}: A

\vspace{\baselineskip}\textbf{Dica}: Utilize o teorema de Pitágoras para encontrar o cateto desconhecido antes de calcular a tangente.

\paragraph{Questão 23:}
{Para um triângulo retângulo onde a hipotenusa vale 5d e um dos catetos vale 3d, a tangente do ângulo oposto ao cateto menor é:}

\vspace{\baselineskip}Opções:

\vspace{\baselineskip}\textbf{A) \frac{3}{4}}: 3/4 

\textbf{B) \frac{4}{3}}: 4/3 

\textbf{C) \frac{4}{5}}: 4/5 

\textbf{D) \frac{3}{5}}: 3/5 

\textbf{E) \frac{5}{4}}: 5/4 

\vspace{\baselineskip}\textbf{Resposta Correta}: C

\vspace{\baselineskip}\textbf{Dica}: Lembre-se de que a tangente é a razão do cateto oposto sobre o adjacente.

\paragraph{Questão 24:}
{Em um triângulo retângulo, se um cateto mede 7e e a hipotenusa mede 25e, calcule a tangente do ângulo oposto ao cateto:}

\vspace{\baselineskip}Opções:

\vspace{\baselineskip}\textbf{A) \frac{7}{24}}: 7/24 

\textbf{B) \frac{24}{7}}: 24/7 

\textbf{C) \frac{25}{7}}: 25/7 

\textbf{D) \frac{7}{25}}: 7/25 

\textbf{E) \frac{24}{25}}: 24/25 

\vspace{\baselineskip}\textbf{Resposta Correta}: A

\vspace{\baselineskip}\textbf{Dica}: Primeiro, encontre o cateto desconhecido usando o teorema de Pitágoras.

\paragraph{Questão 25:}
{Dado um triângulo retângulo onde a hipotenusa tem comprimento 10f e um cateto mede 6f, determine a tangente do ângulo oposto ao outro cateto:}

\vspace{\baselineskip}Opções:

\vspace{\baselineskip}\textbf{A) \frac{6}{8}}: 6/8 

\textbf{B) \frac{8}{6}}: 8/6 

\textbf{C) \frac{3}{5}}: 3/5 

\textbf{D) \frac{4}{5}}: 4/5 

\textbf{E) \frac{5}{3}}: 5/3 

\vspace{\baselineskip}\textbf{Resposta Correta}: B

\vspace{\baselineskip}\textbf{Dica}: A tangente é a relação entre o cateto oposto e o cateto adjacente.

\paragraph{Questão 26:}
{Num triângulo retângulo, se os catetos medem 8g e 15g, qual é a tangente do ângulo oposto ao cateto maior?}

\vspace{\baselineskip}Opções:

\vspace{\baselineskip}\textbf{A) \frac{8}{15}}: 8/15 

\textbf{B) \frac{15}{8}}: 15/8 

\textbf{C) \frac{8}{17}}: 8/17 

\textbf{D) \frac{15}{17}}: 15/17 

\textbf{E) \frac{17}{8}}: 17/8 

\vspace{\baselineskip}\textbf{Resposta Correta}: A

\vspace{\baselineskip}\textbf{Dica}: Lembre-se de calcular a tangente como cateto oposto sobre cateto adjacente.

\paragraph{Questão 27:}
{Considere um triângulo retângulo com hipotenusa 17h e cateto 8h, a tangente do ângulo 
oposto ao cateto desconhecido é:}

\vspace{\baselineskip}Opções:

\vspace{\baselineskip}\textbf{A) \frac{15}{8}}: 15/8 

\textbf{B) \frac{8}{15}}: 8/15 

\textbf{C) \frac{17}{8}}: 17/8 

\textbf{D) \frac{8}{17}}: 8/17 

\textbf{E) \frac{15}{17}}: 15/17 

\vspace{\baselineskip}\textbf{Resposta Correta}: A

\vspace{\baselineskip}\textbf{Dica}: Calcule o outro cateto usando o teorema de Pitágoras antes de encontrar a tangente.

\paragraph{Questão 28:}
{Se em um triângulo retângulo um cateto tem 9i e a hipotenusa mede 41i, então a tangente do ângulo oposto ao cateto 9i é:}

\vspace{\baselineskip}Opções:

\vspace{\baselineskip}\textbf{A) \frac{9}{40}}: 9/40 

\textbf{B) \frac{40}{9}}: 40/9 

\textbf{C) \frac{9}{41}}: 9/41 

\textbf{D) \frac{40}{41}}: 40/41 

\textbf{E) \frac{41}{9}}: 41/9 

\vspace{\baselineskip}\textbf{Resposta Correta}: A

\vspace{\baselineskip}\textbf{Dica}: Encontre o outro cateto usando o teorema de Pitágoras para obter a tangente corretamente.

\paragraph{Questão 29:}
{Em um triângulo retângulo com hipotenusa de 26j e um cateto de 10j, a tangente do ângulo oposto ao cateto desconhecido é:}

\vspace{\baselineskip}Opções:

\vspace{\baselineskip}\textbf{A) \frac{12}{5}}: 12/5 

\textbf{B) \frac{5}{12}}: 5/12 

\textbf{C) \frac{10}{24}}: 10/24 

\textbf{D) \frac{24}{10}}: 24/10 

\textbf{E) \frac{26}{5}}: 26/5 

\vspace{\baselineskip}\textbf{Resposta Correta}: D

\vspace{\baselineskip}\textbf{Dica}: Utilize o teorema de Pitágoras para calcular o cateto desconhecido antes de usar a tangente.

\paragraph{Questão 30:}
{Se em um triângulo retângulo a hipotenusa mede 29k e um dos catetos mede 21k, determine a tangente do ângulo oposto ao cateto menor:}

\vspace{\baselineskip}Opções:

\vspace{\baselineskip}\textbf{A) \frac{21}{20}}: 21/20 

\textbf{B) \frac{20}{21}}: 20/21 

\textbf{C) \frac{20}{29}}: 20/29 

\textbf{D) \frac{21}{29}}: 21/29 

\textbf{E) \frac{29}{20}}: 29/20 

\vspace{\baselineskip}\textbf{Resposta Correta}: B

\vspace{\baselineskip}\textbf{Dica}: Lembre-se de que a tangente é o cateto oposto dividido pelo cateto adjacente.

\paragraph{Questão 31:}
{Em um triângulo retângulo, a cotangente de um de seus ângulos agudos é \( \frac{1}{3} \). Sabendo-se que a hipotenusa desse triângulo é 10, o valor do cosseno desse mesmo ângulo é:}

\vspace{\baselineskip}Opções:

\vspace{\baselineskip}\textbf{A)}: \( \frac{\sqrt{90}}{10} \) 

\textbf{B)}: \( \frac{3\sqrt{10}}{10} \) 

\textbf{C)}: \( \frac{\sqrt{10}}{10} \) 

\textbf{D)}: \( \frac{3}{10} \) 

\textbf{E)}: \( \frac{1}{10} \) 

\vspace{\baselineskip}\textbf{Resposta Correta}: B)

\vspace{\baselineskip}\textbf{Dica}: Use a identidade trigonométrica cotangente \( \theta = \frac{1}{\tan \theta} \) e o teorema de Pitágoras para encontrar o valor desejado.

\paragraph{Questão 32:}
{Em um triângulo retângulo, a secante de um dos ângulos agudos é \( \frac{5}{3} \). Sabendo-se que a hipotenusa do triângulo é 6, o valor do seno desse mesmo ângulo é:}

\vspace{\baselineskip}Opções:

\vspace{\baselineskip}\textbf{A)}: \( \frac{3\sqrt{11}}{6} \) 

\textbf{B)}: \( \frac{3}{6} \) 

\textbf{C)}: \( \frac{\sqrt{11}}{6} \) 

\textbf{D)}: \( \frac{4}{6} \) 

\textbf{E)}: \( \frac{3\sqrt{2}}{6} \) 

\vspace{\baselineskip}\textbf{Resposta Correta}: A)

\vspace{\baselineskip}\textbf{Dica}: Converta a secante em cosseno para calcular o cateto, depois use o teorema de Pitágoras.

\paragraph{Questão 33:}
{Em um triângulo retângulo, a cosecante de um dos ângulos agudos é \( \frac{5}{4} \). Sabendo-se que a hipotenusa mede 5, o valor do cosseno desse ângulo é:}

\vspace{\baselineskip}Opções:

\vspace{\baselineskip}\textbf{A)}: \( \frac{3}{5} \) 

\textbf{B)}: \( \frac{4}{5} \) 

\textbf{C)}: \( \frac{3}{4} \) 

\textbf{D)}: \( \frac{4}{3} \) 

\textbf{E)}: \( \frac{\sqrt{3}}{4} \) 

\vspace{\baselineskip}\textbf{Resposta Correta}: A)

\vspace{\baselineskip}\textbf{Dica}: Lembre-se que a cosecante é a inversa do seno e utilize o teorema de Pitágoras.

\paragraph{Questão 34:}
{Em um triângulo retângulo, a tangente de um dos ângulos agudos é \( \frac{4}{3} \). Se a hipotenusa do triângulo é 10, o valor do seno desse ângulo é:}

\vspace{\baselineskip}Opções:

\vspace{\baselineskip}\textbf{A)}: \( \frac{4}{5} \) 

\textbf{B)}: \( \frac{3}{5} \) 

\textbf{C)}: \( \frac{5}{10} \) 

\textbf{D)}: \( \frac{\sqrt{3}}{10} \) 

\textbf{E)}: \( \frac{8}{10} \) 

\vspace{\baselineskip}\textbf{Resposta Correta}: A)

\vspace{\baselineskip}\textbf{Dica}: Use a identidade da tangente e o teorema de Pitágoras para encontrar o valor do cateto oposto.

\paragraph{Questão 35:}
{Em um triângulo retângulo, a cotangente de um dos ângulos agudos é \( \frac{3}{2} \). Sabendo-se que a hipotenusa do triângulo é 13, o valor do seno desse ângulo é:}

\vspace{\baselineskip}Opções:

\vspace{\baselineskip}\textbf{A)}: \( \frac{3}{5} \) 

\textbf{B)}: \( \frac{4}{5} \) 

\textbf{C)}: \( \frac{12}{13} \) 

\textbf{D)}: \( \frac{5}{13} \) 

\textbf{E)}: \( \frac{10}{13} \) 

\vspace{\baselineskip}\textbf{Resposta Correta}: E)

\vspace{\baselineskip}\textbf{Dica}: Converta a cotangente para obter a tangente e depois use o teorema de Pitágoras para o cateto oposto.

\paragraph{Questão 36:}
{Em um triângulo retângulo, a secante de um dos ângulos agudos é \( \frac{13}{12} \). Sabendo-se que a hipotenusa do triângulo é 15, o valor do cosseno desse ângulo é:}

\vspace{\baselineskip}Opções:

\vspace{\baselineskip}\textbf{A)}: \( \frac{12}{13} \) 

\textbf{B)}: \( \frac{13}{15} \) 

\textbf{C)}: \( \frac{3}{12} \) 

\textbf{D)}: \( \frac{5}{13} \) 

\textbf{E)}: \( \frac{12}{15} \) 

\vspace{\baselineskip}\textbf{Resposta Correta}: A)

\vspace{\baselineskip}\textbf{Dica}: Lembre-se que a secante é o inverso do cosseno.

\paragraph{Questão 37:}
{Em um triângulo retângulo, a tangente de um dos ângulos agudos é \( \frac{5}{12} \). Se a hipotenusa do triângulo é 13, o valor do cosseno desse ângulo é:}

\vspace{\baselineskip}Opções:

\vspace{\baselineskip}\textbf{A)}: \( \frac{12}{13} \) 

\textbf{B)}: \( \frac{5}{13} \) 

\textbf{C)}: \( \frac{\sqrt{119}}{13} \) 

\textbf{D)}: \( \frac{13}{5} \) 

\textbf{E)}: \( \frac{\sqrt{5}}{13} \) 

\vspace{\baselineskip}\textbf{Resposta Correta}: A)

\vspace{\baselineskip}\textbf{Dica}: Use a razão da tangente para encontrar o cálculo do cateto adjacente e verifique com o teorema de Pitágoras.

\paragraph{Questão 38:}
{Em um triângulo retângulo, a cosecante de um dos ângulos agudos é \( \frac{13}{5} \). Sabendo-se que a hipotenusa mede 13, o valor do seno desse ângulo é:}

\vspace{\baselineskip}Opções:

\vspace{\baselineskip}\textbf{A)}: \( \frac{5}{13} \) 

\textbf{B)}: \( \frac{12}{13} \) 

\textbf{C)}: \( \frac{13}{5} \) 

\textbf{D)}: \( \frac{5}{12} \) 

\textbf{E)}: \( \frac{\sqrt{13}}{5} \) 

\vspace{\baselineskip}\textbf{Resposta Correta}: A)

\vspace{\baselineskip}\textbf{Dica}: A cosecante é a inversa do seno. Use essa relação para encontrar o valor correto.

\paragraph{Questão 39:}
{Em um triângulo retângulo, a secante de um dos ângulos agudos é \( \frac{10}{8} \). Sabendo-se que a hipotenusa do triângulo é 6, o valor do cosseno desse ângulo é:}

\vspace{\baselineskip}Opções:

\vspace{\baselineskip}\textbf{A)}: \( \frac{8}{10} \) 

\textbf{B)}: \( \frac{4}{5} \) 

\textbf{C)}: \( \frac{3}{5} \) 

\textbf{D)}: \( \frac{6}{10} \) 

\textbf{E)}: \( \frac{8}{10} \) 

\vspace{\baselineskip}\textbf{Resposta Correta}: B)

\vspace{\baselineskip}\textbf{Dica}: A secante é o inverso do cosseno, facilitando encontrar o valor correto.

\paragraph{Questão 40:}
{Em um triângulo retângulo, a tangente de um dos ângulos agudos é \( \frac{7}{24} \). Se a hipotenusa do triângulo é de 25, o valor do cosseno desse ângulo é:}

\vspace{\baselineskip}Opções:

\vspace{\baselineskip}\textbf{A)}: \( \frac{24}{25} \) 

\textbf{B)}: \( \frac{7}{25} \) 

\textbf{C)}: \( \frac{\sqrt{576}}{25} \) 

\textbf{D)}: \( \frac{14}{25} \) 

\textbf{E)}: \( \frac{25}{7} \) 

\vspace{\baselineskip}\textbf{Resposta Correta}: A)

\vspace{\baselineskip}\textbf{Dica}: Use a razão do ângulo para calcular o cateto adjacente e aplique o teorema de Pitágoras.

\paragraph{Questão 41:}
{Uma passarela é construída ligando dois vértices opostos de um campo retangular com dimensões de 40 m e 30 m. Qual o valor do cosseno do ângulo formado entre a passarela e o lado menor do campo?}

\vspace{\baselineskip}Opções:

\vspace{\baselineskip}\textbf{A}: 3/5 

\textbf{B}: 4/5 

\textbf{C}: 1/5 

\textbf{D}: 2/5 

\textbf{E}: 5/4 

\vspace{\baselineskip}\textbf{Resposta Correta}: A

\vspace{\baselineskip}\textbf{Dica}: Lembre-se de que o cosseno do ângulo é o cateto adjacente dividido pela hipotenusa.

\paragraph{Questão 42:}
{Para melhorar a drenagem de um lote de 60 m por 45 m, foi instalado um tubo que liga vértices opostos do lote. Qual a razão entre a hipotenusa e o lado de maior medida do lote?}

\vspace{\baselineskip}Opções:

\vspace{\baselineskip}\textbf{A}: 4/5 

\textbf{B}: 3/4 

\textbf{C}: 5/9 

\textbf{D}: 7/9 

\textbf{E}: 8/10 

\vspace{\baselineskip}\textbf{Resposta Correta}: A

\vspace{\baselineskip}\textbf{Dica}: Considere o triângulo retângulo formado pela diagonal e os lados do lote.

\paragraph{Questão 43:}
{Uma ponte é construída sobre um rio de largura 50 m e comprimento 120 m, ligando dois vértices opostos. Determine o valor do seno do ângulo formado entre a ponte e o lado de comprimento maior.}

\vspace{\baselineskip}Opções:

\vspace{\baselineskip}\textbf{A}: 1/3 

\textbf{B}: 3/13 

\textbf{C}: 5/13 

\textbf{D}: 12/13 

\textbf{E}: 13/5 

\vspace{\baselineskip}\textbf{Resposta Correta}: C

\vspace{\baselineskip}\textbf{Dica}: O seno é a razão entre o cateto oposto e a hipotenusa.

\paragraph{Questão 44:}
{Em um parque retangular de 36 m x 48 m, é instalado um cabo do vértice superior esquerdo ao vértice inferior direito. Calcule o valor do cosseno do ângulo entre o cabo e o lado maior do terreno.}

\vspace{\baselineskip}Opções:

\vspace{\baselineskip}\textbf{A}: 3/5 

\textbf{B}: 4/5 

\textbf{C}: 5/3 

\textbf{D}: 5/4 

\textbf{E}: 7/8 

\vspace{\baselineskip}\textbf{Resposta Correta}: B

\vspace{\baselineskip}\textbf{Dica}: Relembre que cosseno é cateto adjacente sobre hipotenusa.

\paragraph{Questão 45:}
{Em um campo de futebol com dimensões 64 m e 48 m, instala-se um sistema de irrigação em formato de tubo que atravessa diagonalmente o campo. Qual a tangente do ângulo que o tubo faz com o lado menor do campo?}

\vspace{\baselineskip}Opções:

\vspace{\baselineskip}\textbf{A}: 3/4 

\textbf{B}: 4/3 

\textbf{C}: 3/5 

\textbf{D}: 5/4 

\textbf{E}: 5/3 

\vspace{\baselineskip}\textbf{Resposta Correta}: B

\vspace{\baselineskip}\textbf{Dica}: Tangente do ângulo é a razão entre o cateto oposto e o cateto adjacente.

\paragraph{Questão 46:}
{Um telhado triangular é construído com um lado retangular de 28 m por 21 m e usa uma viga que atravessa o triângulo pela diagonal. Qual a razão entre a hipotenusa e o lado maior do telhado?}

\vspace{\baselineskip}Opções:

\vspace{\baselineskip}\textbf{A}: 2/3 

\textbf{B}: 5/6 

\textbf{C}: 3/4 

\textbf{D}: 7/10 

\textbf{E}: 4/5 

\vspace{\baselineskip}\textbf{Resposta Correta}: E

\vspace{\baselineskip}\textbf{Dica}: Considere o triângulo retângulo formado e relacione hipotenusa e catetos.

\paragraph{Questão 47:}
{Em um estacionamento retangular de 50 m x 20 m, foi instalada uma linha de energia que atravessa diagonalmente o terreno. Determine o valor do seno do ângulo entre a linha e o lado maior do estacionamento.}

\vspace{\baselineskip}Opções:

\vspace{\baselineskip}\textbf{A}: 2/5 

\textbf{B}: 3/7 

\textbf{C}: 5/6 

\textbf{D}: 4/5 

\textbf{E}: 1/3 

\vspace{\baselineskip}\textbf{Resposta Correta}: E

\vspace{\baselineskip}\textbf{Dica}: Seno do ângulo é a razão do cateto oposto pela hipotenusa.

\paragraph{Questão 48:}
{Durante a construção de um campo de futebol retangular de dimensões 80 m e 60 m, a diagonal foi utilizada como referência. Qual o valor do cosseno do ângulo formado entre a diagonal e o lado maior?}

\vspace{\baselineskip}Opções:

\vspace{\baselineskip}\textbf{A}: 1/5 

\textbf{B}: 3/4 

\textbf{C}: 4/5 

\textbf{D}: 5/6 

\textbf{E}: 5/8 

\vspace{\baselineskip}\textbf{Resposta Correta}: C

\vspace{\baselineskip}\textbf{Dica}: Faça uso da definição de cosseno em um triângulo retângulo.

\paragraph{Questão 49:}
{Um gramado retangular de 90 m por 30 m possui uma trilha em linha reta, ligando dois vértices opostos. Calcule a tangente do ângulo formado entre a trilha e o lado menor do gramado.}

\vspace{\baselineskip}Opções:

\vspace{\baselineskip}\textbf{A}: 3/10 

\textbf{B}: 1/3 

\textbf{C}: 7/10 

\textbf{D}: 9/10 

\textbf{E}: 2/3 

\vspace{\baselineskip}\textbf{Resposta Correta}: B

\vspace{\baselineskip}\textbf{Dica}: Tangente é a relação entre cateto oposto e cateto adjacente.

\paragraph{Questão 50:}
{Um carpinteiro decidiu passar um fio através de um terreno retangular de medidas 18 m e 24 m unindo dois vértices opostos. Qual a razão entre o fio e o lado de menor medida?}

\vspace{\baselineskip}Opções:

\vspace{\baselineskip}\textbf{A}: 4/5 

\textbf{B}: 3/2 

\textbf{C}: 5/3 

\textbf{D}: 5/4 

\textbf{E}: 6/5 

\vspace{\baselineskip}\textbf{Resposta Correta}: A

\vspace{\baselineskip}\textbf{Dica}: Calcule a hipotenusa usando o Teorema de Pitágoras primeiro.

\paragraph{Questão 51:}
{Um avião parte do ponto X e voa 30 km em linha reta até o ponto Y, formando um ângulo de 30° com a linha reta que liga o ponto X a um ponto Z distante 20 km de X. Qual é a distância do ponto Y até Z, em linha reta, em quilômetros?}

\vspace{\baselineskip}Opções:

\vspace{\baselineskip}\textbf{A}: 15√3 km 

\textbf{B}: 10√3 km 

\textbf{C}: 20√3 km 

\textbf{D}: 25√3 km 

\textbf{E}: 35√3 km 

\vspace{\baselineskip}\textbf{Resposta Correta}: B

\vspace{\baselineskip}\textbf{Dica}: Utilize as propriedades dos triângulos notáveis e identifique as relações entre os lados do triângulo em questão.

\paragraph{Questão 52:}
{Um paralelepípedo retângulo tem base quadrada e mede 10 cm de altura. Um ponto P está sobre um lado da base e equidista das duas diagonais da base. Qual é a distância do ponto P ao vértice superior oposto do paralelepípedo?}

\vspace{\baselineskip}Opções:

\vspace{\baselineskip}\textbf{A}: √104 cm 

\textbf{B}: √100 cm 

\textbf{C}: √95 cm 

\textbf{D}: √85 cm 

\textbf{E}: √75 cm 

\vspace{\baselineskip}\textbf{Resposta Correta}: B

\vspace{\baselineskip}\textbf{Dica}: Considere a propriedade da base quadrada e aplique o teorema de Pitágoras nas diagonais e altura.

\paragraph{Questão 53:}
{Uma linha férrea retilínea A está a 5 km de uma estação, formando um ângulo de 60° com a linha férrea retilínea B. Um trilho C, perpendicular a B, parte da estação. Qual é a distância da estação até a linha férrea A através do trilho C, em quilômetros?}

\vspace{\baselineskip}Opções:

\vspace{\baselineskip}\textbf{A}: 5√3 km 

\textbf{B}: 2.5 km 

\textbf{C}: 5/√3 km 

\textbf{D}: 5 km 

\textbf{E}: 5/2 km 

\vspace{\baselineskip}\textbf{Resposta Correta}: A

\vspace{\baselineskip}\textbf{Dica}: Desenhe o triângulo formado e use a relação das projeções para calcular a distância.

\paragraph{Questão 54:}
{Uma montanha possui uma inclinação de 30° em relação ao solo plano. Um túnel é cavado perpendicularmente ao solo plano, atravessando a montanha. Se a base do túnel inicia a 3 km do topo, qual é o comprimento do túnel perfurado?}

\vspace{\baselineskip}Opções:

\vspace{\baselineskip}\textbf{A}: 6 km 

\textbf{B}: 3√3 km 

\textbf{C}: 9 km 

\textbf{D}: 3 km 

\textbf{E}: 3/2 km 

\vspace{\baselineskip}\textbf{Resposta Correta}: B

\vspace{\baselineskip}\textbf{Dica}: Considere a relação dos ângulos na projeção ortogonal para calcular o comprimento necessário.

\paragraph{Questão 55:}
{Um rio flui em linha reta entre duas cidades A e B, distantes 12 km uma da outra. Uma ponte reta é introduzida perpendicularmente ao fluxo, a 3 km de A. Qual é a distância total do fluxo de água percorrida pela ponte?}

\vspace{\baselineskip}Opções:

\vspace{\baselineskip}\textbf{A}: 6 km 

\textbf{B}: 8 km 

\textbf{C}: 5 km 

\textbf{D}: 9 km 

\textbf{E}: 4 km 

\vspace{\baselineskip}\textbf{Resposta Correta}: A

\vspace{\baselineskip}\textbf{Dica}: Desenhe o triângulo formado pela ponte e os caminhos fornecidos, aplicando propriedades geométricas.

\paragraph{Questão 56:}
{Uma avenida linear A encontra o começo de outra avenida B em um ângulo de 120°. Há uma pista C perpendicular a A que corta a avenida B a 6 km do ponto de início de B. Qual é a distância entre o ponto de interseção e o ponto no final da avenida A, dado que A tem 5 km de comprimento?}

\vspace{\baselineskip}Opções:

\vspace{\baselineskip}\textbf{A}: 5√3 km 

\textbf{B}: 7 km 

\textbf{C}: 3 km 

\textbf{D}: 5 km 

\textbf{E}: 3√3 km 

\vspace{\baselineskip}\textbf{Resposta Correta}: E

\vspace{\baselineskip}\textbf{Dica}: Utilize relações trigonométricas e as propriedades dos ângulos formados para determinar a distância do ponto.

\paragraph{Questão 57:}
{Um grupo de árvores forma um ângulo de 45° ao longo de uma colina de 5 km. Uma trilha projetada paralela à base da colina é perpendicular à linha das árvores, partindo do pé da colina. Qual é a extensão total da trilha percorrida entre a base e o pico das árvores?}

\vspace{\baselineskip}Opções:

\vspace{\baselineskip}\textbf{A}: 5√2 km 

\textbf{B}: 5/√2 km 

\textbf{C}: 10 km 

\textbf{D}: 5 km 

\textbf{E}: 2.5 km 

\vspace{\baselineskip}\textbf{Resposta Correta}: D

\vspace{\baselineskip}\textbf{Dica}: Use as definições de ângulos complementares e relações entre os lados do triângulo.

\paragraph{Questão 58:}
{Dois postes de eletricidade estão 8 km afastados, com fio retilíneo entre eles formando um ângulo de 37°. Se um fio auxiliar sai perpendicularmente do primeiro poste, qual é a distância horizontal deste fio até interceptar a linha administrada ao segundo poste?}

\vspace{\baselineskip}Opções:

\vspace{\baselineskip}\textbf{A}: 6 km 

\textbf{B}: 8 km 

\textbf{C}: 10 km 

\textbf{D}: 4 km 

\textbf{E}: 5 km 

\vspace{\baselineskip}\textbf{Resposta Correta}: A

\vspace{\baselineskip}\textbf{Dica}: Projete a situação em um plano cartesiano e aplique relações geométricas para encontrar a distância correta.

\paragraph{Questão 59:}
{Uma ponte de 200 m atravessa um rio em um ângulo de 60° com a margem. Há um caminho paralelo à margem que intercepta a ponte no meio. Qual é a altura da margem ao ponto médio da ponte?}

\vspace{\baselineskip}Opções:

\vspace{\baselineskip}\textbf{A}: 100 m 

\textbf{B}: 50 m 

\textbf{C}: 75 m 

\textbf{D}: 125 m 

\textbf{E}: 150 m 

\vspace{\baselineskip}\textbf{Resposta Correta}: C

\vspace{\baselineskip}\textbf{Dica}: Projete o triângulo com a base e uso das relações trigonométricas para calcular a altura perpendicular.

\paragraph{Questão 60:}
{Um navio navega da ilha X por 60 km em direção nordeste. Chegando no ponto Y, forma um triângulo retângulo com uma ilha Z que está 30 km ao leste de X. Qual é a distância em linha reta de Y até Z?}

\vspace{\baselineskip}Opções:

\vspace{\baselineskip}\textbf{A}: 30√2 km 

\textbf{B}: 30 km 

\textbf{C}: 60 km 

\textbf{D}: 90 km 

\textbf{E}: 45 km 

\vspace{\baselineskip}\textbf{Resposta Correta}: A

\vspace{\baselineskip}\textbf{Dica}: Considere a condição do triângulo e a utilização do teorema de Pitágoras para obter sua solução.

\end{document}